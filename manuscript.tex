%%
%% Copyright 2007-2020 Elsevier Ltd
%%
%% This file is part of the 'Elsarticle Bundle'.
%% ---------------------------------------------
%%
%% It may be distributed under the conditions of the LaTeX Project Public
%% License, either version 1.2 of this license or (at your option) any
%% later version.  The latest version of this license is in
%%    http://www.latex-project.org/lppl.txt
%% and version 1.2 or later is part of all distributions of LaTeX
%% version 1999/12/01 or later.
%%
%% The list of all files belonging to the 'Elsarticle Bundle' is
%% given in the file `manifest.txt'.
%%

%% Template article for Elsevier's document class `elsarticle'
%% with numbered style bibliographic references
%% SP 2008/03/01
%%
%%
%%
%% $Id: elsarticle-template-num.tex 190 2020-11-23 11:12:32Z rishi $
%%
%%
\documentclass[preprint,12pt]{elsarticle}

%% Use the option review to obtain double line spacing
%% \documentclass[authoryear,preprint,review,12pt]{elsarticle}

%% Use the options 1p,twocolumn; 3p; 3p,twocolumn; 5p; or 5p,twocolumn
%% for a journal layout:
%% \documentclass[final,1p,times]{elsarticle}
%% \documentclass[final,1p,times,twocolumn]{elsarticle}
%% \documentclass[final,3p,times]{elsarticle}
%% \documentclass[final,3p,times,twocolumn]{elsarticle}
%% \documentclass[final,5p,times]{elsarticle}
%% \documentclass[final,5p,times,twocolumn]{elsarticle}

%% For including figures, graphicx.sty has been loaded in
%% elsarticle.cls. If you prefer to use the old commands
%% please give \usepackage{epsfig}

%% The amssymb package provides various useful mathematical symbols
\usepackage{amssymb}
%% The amsthm package provides extended theorem environments
%% \usepackage{amsthm}
%% The amsmath package provides a handful of options for displaying equations.
\usepackage{amsmath}

%% The lineno packages adds line numbers. Start line numbering with
%% \begin{linenumbers}, end it with \end{linenumbers}. Or switch it on
%% for the whole article with \linenumbers.
\usepackage{lineno}

\journal{Physics Letters A}

\begin{document}

\begin{frontmatter}

%% Title, authors and addresses

%% use the tnoteref command within \title for footnotes;
%% use the tnotetext command for theassociated footnote;
%% use the fnref command within \author or \address for footnotes;
%% use the fntext command for theassociated footnote;
%% use the corref command within \author for corresponding author footnotes;
%% use the cortext command for theassociated footnote;
%% use the ead command for the email address,
%% and the form \ead[url] for the home page:
%% \title{Title\tnoteref{label1}}
%% \tnotetext[label1]{}
%% \author{Name\corref{cor1}\fnref{label2}}
%% \ead{email address}
%% \ead[url]{home page}
%% \fntext[label2]{}
%% \cortext[cor1]{}
%% \affiliation{organization={},
%%             addressline={},
%%             city={},
%%             postcode={},
%%             state={},
%%             country={}}
%% \fntext[label3]{}

\title{Multiple quantum NMR in solids as a method of determination of Wigner--Yanase skew information}

%% use optional labels to link authors explicitly to addresses:
%% \author[label1,label2]{}
%% \affiliation[label1]{organization={},
%%             addressline={},
%%             city={},
%%             postcode={},
%%             state={},
%%             country={}}
%%
%% \affiliation[label2]{organization={},
%%             addressline={},
%%             city={},
%%             postcode={},
%%             state={},
%%             country={}}




\author[icp]{S.~I.~Doronin}
\author[icp]{E.~B.~Fel'dman}
\author[icp,msu]{I.~D.~Lazarev}
\address[icp]{Institute of Problems of Chemical Physics of Russian Academy of Sciences, \\ Chernogolovka, Moscow Region, Russia 142432}
\address[msu]{Faculty of Fundamental Physical-Chemical Engineering, Lomonosov Moscow State University, GSP-1, Moscow, Russia 119991}

\begin{abstract}
%% Text of abstract
A connection of the Wigner--Yanase skew information and multiple quantum (MQ) NMR coherences is considered at different temperatures and evolution times of nuclear spins with the dipole-dipole interactions in MQ NMR experiments in solids.
It is shown that the Wigner--Yanase skew information at temperature $T$ is equal to the double second moment of the MQ NMR spectrum at the temperature twice large and any evolution times.
A comparison of the many--spin entanglement obtained with the Wigner--Yanase information and the Fisher information is conducted.
\end{abstract}

%%Graphical abstract
%\begin{graphicalabstract}
%\includegraphics{...}
%\end{graphicalabstract}

%%Research highlights
\begin{highlights}
	\item multiple quantum NMR spectroscopy
	\item connection between the Wigner-Yanase skew information and the second moment of the distribution of multiple quantum coherences
	\item the dependence of the number of entangled spins on temperature
\end{highlights}

\begin{keyword}
%% keywords here, in the form: keyword \sep keyword
	many-spin entanglement \sep
	Fisher information \sep
	Wigner-Yanase skew information \sep
	multiple quantum NMR \sep
	multiple quantum coherences \sep
	second moment \sep
	temperature
%% PACS codes here, in the form: \PACS code \sep code
%\PACS 0000 \sep 1111
%%
%% MSC codes here, in the form: \MSC code \sep code
%% or \MSC[2008] code \sep code (2000 is the default)
%\MSC 0000 \sep 1111
\end{keyword}

\end{frontmatter}

\linenumbers

%% main text

\section{Introduction}
\label{sec:1}
The Wigner--Yanase skew information~\cite{1,2,3,4} together with the Fisher information~\cite{5,6} allow to develop the powerful methods for an investigation of entanglement, including the many--particle entanglement~\cite{7,8}.
Further investigations of the many--particle entanglement require a development of the corresponding experimental methods.
In particular, it was shown~\cite{7,9} that a lower bound on the quantum Fisher information~\cite{5,6} coincides with the double second moment of the spectrum of multiple quantum (MQ) coherences.
As a result, the lower bound on the quantum Fisher information can be found in MQ~NMR experiments~\cite{10},
in cold--atom experiments, including Bose--Einstein condensates, ultracold atoms in cavities, and trapped ions~\cite{11,12,13,14,15}.
Using the properties of the quantum Fisher information one can obtain the number of entangled particles (spins) in the system under consideration~\cite{7}
and even find the dependence of the number of entangled spins on temperature~\cite{9}.


The Wigner--Yanase skew information~\cite{1,2,3,4} is also connected with the spectrum of MQ coherences.
In particular, we demonstrate in the present article that the Wigner--Yanase skew information in a spin system $(s = 1/2)$ with the dipole--dipole interactions (DDI) in MQ NMR experiment~\cite{10} at the system temperature $T$ equals to the double second moment of the MQ NMR spectrum obtained at temperature $2T$.
Using the properties of the Wigner--Yanase skew information one can investigate the many--spin entanglement on the basis of the MQ NMR spectroscopy~\cite{10}.


The main aim of the present article is a development of the method of extracting the Wigner--Yanase skew information from the MQ NMR spectra.
We also compare on the simple models~\cite{8,16} the many--spin entanglement obtained both with the Wigner--Yanase information and the Fisher information.


The article is organized as follows.
In Sec.~\ref{sec:2} a short introduction to the MQ NMR spectroscopy is given.
The connection of the Wigner--Yanase skew information with the second moment of the MQ NMR spectrum is obtained in Sec.~\ref{sec:3}.
A comparison of the many--spin entanglement obtained with the Wigner--Yanase and the Fisher information on the simple models~\cite{8,16} is conducted in Sec.~\ref{sec:4}.
We briefly discuss our results in the concluding Sec.~\ref{sec:5}.


\section{MQ NMR for solving problems of quantum informatics}
\label{sec:2}


\begin{figure}
	\includegraphics[width=0.95\linewidth]{mq-experiment}
	\caption{The basic scheme of the multiple quantum NMR experiment.}
	\label{fig:1}
\end{figure}

MQ NMR methods are widely used for solving problems of quantum informatics~\cite{17,18}.
The MQ NMR experiment consists of four distinct periods of time as depicted in Fig.~\ref{fig:1}:
preparation $(\tau)$, evolution $(t_1)$, mixing $(\tau)$, and detection $(t_2)$~\cite{10}.
MQ NMR coherences are created by a periodic multipulse sequence, consisting of
$\pm$x-pulses, irradiating the system of the preparation period~\cite{10}.
If the inverse period of the multipulse sequence significantly exceeds the local dipolar field (in frequency units)~\cite{19}
MQ NMR dynamics can be described by the averaged nonsecular two--spin/two--quantum Hamiltonian~$H_{MQ}$~\cite{20}
%
\begin{equation} \label{eq:1}
        H_{MQ} = H^{(+2)} + H^{(-2)} , \quad
        H^{(\pm 2)} = -\frac{1}{2} \sum_{j<k} D_{jk} I_{j} ^\pm I_k^\pm,
\end{equation}
%
where $D_{jk}$ is the coupling constant between spins $j$ and $k$,
and $I_{j}^+, I_k ^-$ are the raising and lowering operators of spin $j$.
On the mixing period the spin system is irradiated by the multiple sequence with $\pm y$--pulses.
As a result, the averaged nonsecular two-spin/two quantum Hamiltonian on the mixing period equals $(-H_{MQ})$~\cite{10}.


In order to investigate the MQ NMR dynamics of the system on the preparation period~\cite{10} one should find the density matrix $\rho(t)$ by solving the Liouville evolution equation~\cite{19}

\begin{equation}
    \label{eq:2}
        i\frac{d\rho(t)}{dt} = [H_{MQ}, \rho(t)]
\end{equation}
with the initial thermodynamic equilibrium density matrix
\begin{equation}
    \label{eq:3}
        \rho(0) = \rho_\mathrm{eq} = \frac{\exp(\frac{\hbar \omega_0}{kT}I_z)}{Z},
\end{equation}
where $Z=Tr \left\{exp\left(\frac{\hbar \omega_0}{kT}I_z\right) \right\}$ is the partition function,
$\hbar$ and $k$ are the Plank and Boltzmann constants respectively,
$\omega_0$ is the Larmor frequency,
$T$ is the temperature,
and $I_z$ is the operator of the projection of the total spin angular momentum on the $z$-axis,
which is directed along the strong external magnetic field.


Following the preparation evolution and mixing periods of the MQ NMR experiments and taking into account the phase increment $\phi$ of the radio-frequency pulses~\cite{10}, the resulting signal $G(\tau,\phi)$ stored as population information is
%
\begin{equation} \label{eq:4}
	\begin{split}
		G(\tau,\phi)
		& = Tr \left\{
			e^{iH_{MQ}\tau} e^{i\phi I_z} e^{-iH_{MQ}\tau} \rho_\mathrm{eq}
			e^{iH_{MQ}\tau} e^{-i\phi I_z} e^{-iH_{MQ}\tau}\rho_\mathrm{eq}
		\right\}
		\\
		& = Tr \left\{
			e^{i\phi I_z} \rho_\mathrm{pre}(\tau,\beta)
      e^{-i\phi I_z}\rho_\mathrm{pre}(\tau,\beta)
		\right\},
	\end{split}
\end{equation}
%
where
%
\begin{equation} \label{eq:5}
	\rho_\mathrm{pre}(\tau,\beta) = e^{-iH_{MQ}\tau}\rho_\mathrm{eq}e^{iH_{MQ}\tau}
\end{equation}
%
is the density matrix at the end of the preparation period,
which can be defined from~Eqs.~(\ref{eq:2},\ref{eq:3}) and $\beta = \frac{\hbar \omega_0}{kT}$.


It is convenient to expand the density matrix $\rho_\mathrm{pre}(\tau, \beta)$ in series as~\cite{21}
\begin{equation}
    \label{eq:6}
        \rho_\mathrm{pre}(\tau,\beta) = \sum_n \rho_{\mathrm{pre},n}(\tau,\beta),
\end{equation}
where $\rho_{\mathrm{pre}, n}(\tau,\beta)$ is the contribution to the density matrix $\rho_\mathrm{pre}(\tau,\beta)$ from the MQ coherence of the n--th order.
Then the resulting signal $G(\tau,\phi)$ of the MQ NMR~\cite{10} can be rewritten as
\begin{equation} \label{eq:7}
    G(\tau, \phi) = \sum_n e^{in\phi}
        Tr\left\{\rho_{\mathrm{pre},n}(\tau,\beta)
        \rho_{\mathrm{pre},-n}(\tau,\beta) \right\},
\end{equation}
where we took into account that
\begin{equation}
    \label{eq:8}
        [I_z, \rho_{\mathrm{pre},n}] = n \rho_{\mathrm{pre},n}
\end{equation}
The normalized intensities of the MQ NMR coherences can be determined as follows
\begin{equation}
    \label{eq:9}
        J_n(\tau,\beta)= \frac{Tr\left\{\rho_{\mathrm{pre},n}(\tau,\beta)
            \rho_{\mathrm{pre},-n}(\tau,\beta)\right\}}
                {Tr(\rho^2_\mathrm{eq})}
\end{equation}
As was shown in~\cite{8},
\begin{equation}
    \label{eq:10}
        Tr(\rho_\mathrm{eq}^2) = \frac{2^N ch^N (\beta)}{Z^2},
\end{equation}
where $N$ is the number of spins.
It was also shown that
\begin{equation}
    \label{eq:11}
        \sum_n J_n(\tau,\beta) = 1
\end{equation}
The second moment (dispersion) $M_2(\tau,\beta)$ of the distribution of the MQ NMR coherences $J_n (\tau,\beta)$ can be calculated from Eq.~(\ref{eq:7}) according to~\cite{22}
\begin{equation}
    \label{eq:12}
        M_2(\tau,\beta) = -\frac{1}{G_2(\tau,0)}
            \frac{d^2 G(t,\beta)}{dt^2}\bigg|
        _{t=0}
\end{equation}
Using Eqs.~(\ref{eq:7},\ref{eq:8},\ref{eq:12}) one can obtain
\begin{equation}
    \label{eq:13}
        M_2 (\tau,\beta) = \sum_n n^2 J_n(\tau,\beta)
\end{equation}
The lower bound on the quantum Fisher information coincides with the double second moment of Eq.~(\ref{eq:13})~\cite{7,9}.
As a result, an analysis of the temperature dependence of the second moment $M_2(\tau,\beta)$ of the distribution of the intensities of the MQ NMR coherences allows us to obtain the number of entangled spins at different temperatures~\cite{8}.
In the following~Sec.~\ref{sec:3} we demonstrate that the Wigner--Yananse skew information is also connected with the second momentum $M_2(\tau,\beta)$
and can be useful for an investigation of the many-spin entanglement.

\section{The Wigner--Yanase skew information and MQ NMR}
\label{sec:3}
The Wigner--Yanase skew information is defined as~\cite{1,2,3}
\begin{equation}
    \label{eq:14}
        I_{WY}(\rho(\tau,\beta),I_z) = -\frac{1}{2}
            Tr([\sqrt{\rho(\tau,\beta)},\sigma_z])^2 =
                -2Tr([\sqrt{\rho(\tau,\beta)},I_z])^2,
\end{equation}
where the Pauli operator $\sigma_z=2I_z$.
Introducing the evolution operator
\begin{equation}
    \label{eq:15}
        V(\tau) = e^{iH_{MQ}\tau}
\end{equation}
and using Eq.~(\ref{eq:3}) one can write the density matrix $\rho(\tau,\beta)$ as follows:
\begin{equation}
    \label{eq:16}
        \rho(\tau,\beta) = V^+(\tau) \frac{e^{\beta I_z}}{Z}V(\tau)
\end{equation}
Now we use the evident relationship:
\begin{equation}
    \label{eq:17}
        \sqrt{\rho(\tau,\beta)} =
            \sqrt{V^+(\tau)\frac{e^{\beta I_z}}{Z}V(\tau)} =
                V^+(\tau) \frac{e^{\frac{\beta}{2}I_z}}{\sqrt{Z}}V(\tau).
\end{equation}
It can be proved by simple calculation:
\begin{equation}
    \label{eq:18}
        \sqrt{\rho}\sqrt{\rho} =
						V^+(\tau)\frac{e^{\frac{\beta}{2}I_z}}{\sqrt{Z}}
                V(\tau)V^+(\tau)\frac{e^{\frac{\beta}{2}I_z}}{\sqrt{Z}}V(\tau) =
						V^+(\tau)\frac{e^{\beta I_z}}{Z}V(\tau) =
        \rho(\tau,\beta)
\end{equation}
%
Then we have
%
\begin{equation} \label{eq:19}
    \left[I_z,\sqrt{\rho(\tau,\beta)}\right]
    = \left[I_z, \sum_k \rho_k \left(\tau, \frac{\beta}{2}\right)\right]
    = \sum_k k\rho_k \left(\tau, \frac{\beta}{2}\right),
\end{equation}
%
and
%
\begin{equation} \label{eq:20}
	Tr\left[I_z,\sqrt{\rho(\tau,\beta)} \right]^2
	= Tr\left\{\sum_{k,k'}kk'
		\rho_k\left(\tau,\frac{\beta}{2}\right)
		\rho_{k'}\left(\tau,\frac{\beta}{2}\right)
	\right\}
	= \sum_k k^2 J_k\left(\tau,\frac{\beta}{2}\right).
\end{equation}
%
Finally, one can obtain that
%
\begin{equation} \label{eq:21}
    I_{WY}\left(\rho(\tau, \beta), I_z\right)
    = 2\sum_k k^2 J_k\left(\tau, \frac{\beta}{2}\right)
    = 2M_2\left(\tau, \frac{\beta}{2}\right)
\end{equation}
%
Thus, we obtain the important assertion.
If the spin system is investigated with MQ NMR at temperature $T\sim\beta^{-1}$ then the Wigner--Yanase skew information equals to the double second momentum of the distribution of the intensities of the MQ NMR coherences at temperature $2T \sim 2\beta^{-1}$ at any moment of the spin evolution.


The Wigner-Yanase skew information is connected with the second moment of the distribution of the MQ NMR coherences analogously to the Fisher information.
We compare these informations in the following Section~\ref{sec:4}.


\section{A comparison of the many-spin entanglement obtained with the Wigner--Yanase information and the Fisher information}
\label{sec:4}

\begin{figure}
	\includegraphics[width=0.95\linewidth]{nanopora_entangled_spins_by_temp}
	\caption{
		The dependence of the number of the entangled spins on the inverse temperature $\beta = \frac{\pi \omega_0}{kT}$;
		black circles - the results are obtained with the Fisher information;
		open circles - the results are obtained with the Wigner--Yanase information.
	}
	\label{fig:2}
\end{figure}

\begin{figure}
	\includegraphics[width=0.95\linewidth]{zigzag_entangled_spins_by_temp}
	\caption{
		The dependence of the number $N_\mathrm{ent}$ of the entangled spins on the parameter $b$ (the inverse temperature) for zigzag chains consisting of six spins.
	}
	\label{fig:3}
\end{figure}

The Wigner--Yanase skew information $I_{WY}(\rho(\tau,\beta),I_z)$ and the Fisher information $I_F(\rho(\tau,\beta),I_z)$ can be used for an investigation of the many-spin entanglement.
Really, it is known~\cite{5,6} that if
$I_{WY}\left( \rho(\tau, \beta), I_z \right)$
or
$I_{F}\left( \rho(\tau, \beta), I_z \right)$
exceeds $mk^2 + (N - mk)^2$,
where $k, m$ are integer and $m$ is the integer part of $N/k$,
than we have $N_\mathrm{ent} = (k + 1) $~--~particle entangled spins in the system.
The informations are connected by the following restriction~\cite{3}
%
\begin{equation} \label{eq:22}
    I_{WY}\left(\rho(\tau,\beta), I_z\right)
    \leq I_F\left(\rho(\tau,\beta), I_z\right)
    \leq 2I_{WY}\left(\rho(\tau,\beta), I_z\right).
\end{equation}
%
The restrictions (22) allow us to hope that the obtained results of the number of entangled spins are not strongly different.
For the comparison we used the model~\cite{23} of a nonspherical nanopore filled with a gas of spin-carrying atoms (for example, xenon) or molecules in a strong external magnetic field.
This model allows to investigate the many-spin entanglement in the spin system consisting of hundreds of nuclear spins~\cite{8}.


We investigated the many-spin entanglement in the spin system, consisting of 201 spins, in a nanopore both with
the Wigner--Yanase information $I_{WY}\left(\rho(\tau, \beta), I_z\right)$
and the Fisher information $I_F\left(\rho(\tau,\beta),I_z\right)$.
In Fig.~\ref{fig:2} the dependence of the number of entangled spins on the inverse temperature is presented.
Fig.~\ref{fig:2} demonstrates that the number of the entangled spins increases when the temperature decreases both for the Wigner--Yanase information and the Fisher information.


The analogous investigation was conducted on the model of the proton zigzag chain in a single crystal of hambergite~\cite{16,24}.
In Fig.~\ref{fig:3} the close results on the many--spin entanglement are presented for the system consisting of six spins at different temperatures for both used informations.


\section{Conclusion}
\label{sec:5}

We studied the connection of the Wigner--Yanase skew information with the second moment of the distribution of the intensities of MQ coherences in the MQ NMR experiment.
It was shown that the Wigner--Yanase skew information at temperature $T \sim \beta^{-1}$ equals to the double second momentum of the MQ NMR spectrum at temperature $2T \sim 2\beta^{-1}$.
We compare also the results on the many--spin entanglement obtained with the Wigner--Yanase skew information and the Fisher information.


\section{Acknowledgement}
\label{sec:6}
We acknowledge funding from the Ministry of Science and Higher Education of the Russian Federation (Grant No.~075-15-2020-779).





%% The Appendices part is started with the command \appendix;
%% appendix sections are then done as normal sections
%\appendix
%\section{Sample Appendix Section}
%\label{sec:sample:appendix}
%Lorem ipsum dolor sit amet...

%% If you have bibdatabase file and want bibtex to generate the
%% bibitems, please use
%%
% \bibliographystyle{elsarticle-num}
% \bibliography{bibliography}

%% else use the following coding to input the bibitems directly in the
%% TeX file.

% \begin{thebibliography}{00}

% %% \bibitem{label}
% %% Text of bibliographic item

% \bibitem{}

% \end{thebibliography}

\begin{thebibliography}{20}
	\bibitem{1} E.P.Wigner, M.M. Yanase, Proc.Nat.Acad.Sei. USA, \textbf{49}, 910-918 (1963)
	\bibitem{2} S.Luo, Phys.Rev.Lett. \textbf{91}, 180403 (2003)
  \bibitem{3} S.Luo, Proc.Amer. Math.Soc. \textbf{132}, No.885-890 (2003)
	\bibitem{4} Z.Chen, Phys.Rev.A. \textbf{71}, 052302 (2005)
  \bibitem{5} G.Toth, I.Apellaniz, J. Phys. A. \textbf{47},424006 (2014)
  \bibitem{6} L.Pezze, A.Smerzi, M.K. Oberthaler, R. Schmied, P.Treutlein, Rev.Mod.Phys. \textbf{90}, 035005 (2018)
  \bibitem{7} M.Gartner, P.Hauke, A.M.Rey, Phys.Rev.Lett. \textbf{120}, 040402 (2018)
  \bibitem{8} S.I.Doronin, E.B.Fel'dman, I.D.Lazarev, Phys. Rev. A. \textbf{100},022330 (2019)
  \bibitem{9} D.Girolami, B.Yadin,Entropy \textbf{19},124 (2017)
  \bibitem{10} J.Baum, M.Munowitz, A.N.Garroway, A.Pines,J.Chem.Phys. \textbf{83}, 2015 (1985)
  \bibitem{11} B. Swingle, G. Bentsen, M. Scheleier-Smith, P. Hayden, Phys.	Rev. A. \textbf{94}, 040302 (2010)
  \bibitem{12} F.M. Cucchietti, J.Opt.Soc. Am. B\textbf{27}, A30 (2010)
  \bibitem{13} I.D. Leroux, M.H. Schleier-Smith, V.Vuletic, Phys.Rev.Lett. \textbf{104}, 073602 (2010)
  \bibitem{14} T.Macri, A. Smerzi, L.Pezze, Phys.Rev. A. \textbf{94}, 010102 (2016)
  \bibitem{15} M.Gartner, J.G.Bohnet, A.Safavi-Naini, M.L. Wall, J.J. Bollinger, A.M.Rey, Nat.Phys. \textbf{13} ,781 (2017)
  \bibitem{16} G.A.Bochkin, S.G.Vasil'ev, S.I. Doronin, E.I.Kuznetsova,I.D.Lazarev,E.B.Fel'dman, Appl.Magn.Reson. \textbf{51},667-678 (2020)
  \bibitem{17} E.B.Fel'dman, A.N. Pyrkov, A.I.Zenchuk, Philos. Trans. R. Soc. London A\textbf{370}, 4690 (2012)
  \bibitem{18} G.B.Furman, V.M.Meerovich, V.L.Sokolovsky, Phys. Rev. A.\textbf{78} 042301 (2008)
  \bibitem{19} M.Goldman. Spin temperature and nuclear magnetic resonance in solids, Oxford,UK, Clarendon Press. 1970.
  \bibitem{20} S.I.Doronin, I.I.Maksimov, E.B.Fel’dman, J. Exp. Theor. Phys.\textbf{91}, 597 (2000)
  \bibitem{21} E.B.Fel'dman, S.Lacelle,Chem.Phys.Lett. \textbf{253}, 27 (1996)
  \bibitem{22} A.Abragam, The Principles of Nuclear Magnetism, Clarendon. Oxford. 1961
  \bibitem{23} J. Baugh, A. Kleinhammes, D. Han, Q. Wang, Y.Wu, Science \textbf{294}, 1505 (2001)
  \bibitem{24} G.A. Bochkin, E.B. Fel'dman, E.B. Kuznetsova, I.D. Lasarev, S.G. Vasil'ev, V.I.Volkov, J. Magn.Reson. \textbf{319}, 106816 (2020)
\end{thebibliography}

\end{document}
\endinput
